%%%%%%%%%%%%%%%%%%%%%%foreword.tex%%%%%%%%%%%%%%%%%%%%%%%%%%%%%%%%%
% sample foreword
%
% Use this file as a template for your own input.
%
%%%%%%%%%%%%%%%%%%%%%%%% Springer %%%%%%%%%%%%%%%%%%%%%%%%%%

\foreword

\setcounter{page}{7}

%% Please have the foreword written here
%Use the template \textit{foreword.tex} together with the document class SVMono (monograph-type books) or SVMult (edited books) to style your foreword\index{foreword}. 

%The foreword covers introductory remarks preceding the text of a book that are written by a \textit{person other than the author or editor} of the book. If applicable, the foreword precedes the preface which is written by the author or editor of the book.


Neuroscientists like to remind us that the brain is the most complex
object in the known universe. The complexity they talk about is
related to the brain's strange ability to integrate sensory inputs, to
learn, to think, to store memories, to develop feeling, and to perform
higher cognitive functions such as consciousness, self-awareness,
mathematics, and yes, being able to write poems and equations about
itself. Mostly, neuroscientists think about complexity in terms of
signal processing and information transfer for which they have
accumulated, through a century of exploration, an encyclopedic
knowledge. Despite these Herculean efforts, much about the brain
remains a mystery. In particular, there is another level of complexity
associated with the brain that has been mostly neglected in the
traditional neurosciences. The brain is a living organ that relies on
a myriad of biological, chemical, and physical processes perfectly
orchestrated to maintain its basic activities. Viewed from a physical
perspective, what makes the brain so fascinating and so complicated as
an organ is that it operates across multiple scales and constantly
uses multiple physical fields. Indeed, processes that take place in
the blink of an eye may be coupled with events that develop over a
lifetime. In space, what happens inside a single neuron may trigger a
global response at the organ or even body level. This large time and
space scale-coupling prevents us from using the physical
scale-separation paradigm that has been so successful in the study of
planets and atoms. Similarly, the brain is a strange composite in
which fluid and soft solid flow into one another, it is a soup of ions
and electrolytes that needs to be carefully balanced at all times, and
is a constant electric and magnetic field generator. This delicate
symphony of processes is what allows the brain to function in
harmony. Any defects or disturbance may lead to severe pathology as
seen in development, trauma, or dementia. To make matter worse, our
thick skull has impaired our ability to probe the brain properly and
even some of its basic defining features, such as its material
response under poking, are poorly understood.

Yet, for the last decade, many scientists from different fields have
come together to rethink the way we think about the brain by adapting
various theories and ideas from engineering, physics, mathematics, and
computer science. To elevate brain modeling to a quantitative physical
theory, one must combine data, experiments, theory, and
computation. For many years, data was a true bottleneck as recording
any physical fields in the living brain was particularly difficult and
invasive. This situation changed completely with the advent of
magnetic resonance imaging (MRI) that became routine in the late
1990s. The basic MRI and its multiple variants and generalizations
have completely, but quietly, revolutionized medical practice by
imparting a reliable, reasonably-high resolution, non-invasive means
of observing the internal states of the brain.  MRI imaging has become
a basic source of information for clinical neuroscience and
neurodegenerative disease research as it allows to map, and measure
properties of the cortex and white matter, to determine the patterns
of water flow within the brain, and to isolate regions of high
cognitive processes.

In another realm of science, completely disconnected from
neurosciences, another quiet revolution was taking place in the same
period. With the rise of computing power, the ability to model and
simulate the response of large three-dimensional structures through
finite-element modeling also became routine. Initially, these methods
were used to evaluate the safety of a bridge or to test the response
of automotive pieces under loads. But, rapidly scientists realized
that they could adapt these ideas to biology by simulating the
response of arteries, heart, lungs, bones, and, eventually, the
brain. The problem was not just to run existing codes to new soft
structures but develop a new mathematical theory of soft biological
materials.  Indeed, with its extreme softness, viscous, active,
nonlinear properties, and composite composition, the brain is not just
a very soft piece of rubber but a complex material with fascinating
properties not shared with any other organ.

Scientists interested in modeling the brain are now in an interesting
situation: sitting on a giant heap of MRI data, with sophisticated
theoretical and computational tools to simulate the brain, they have
to find a way to bridge data to simulation and create a framework
where systematic exploration of scientific and medical questions can
be performed. This is where this little monograph comes in.  This
text, authored by four leading experts in the field, offers an
explicit bridge linking MRI images to scientific computing and
mathematical modeling of the brain.  The authors introduce in simple
and clear terms most of the concepts needed and provide a
freely-available, open-source, and easy-to-use Python software tool
allowing MRI images to be easily transformed into
physiologically-accurate computational assets.  They showcase their
approach by showing how an anisotropic diffusion problem can be solved
using a detailed computational domain, and diffusion tensor,
constructed from a single patient MRI data set. Remarkably, what would
have been a major research project a couple years ago can now be
performed elegantly through their pipeline by any interested reader.

As Wittgenstein wrote in \textit{Philosophical Investigations}: ``We
talk of processes and states, and leave their nature
undecided. Sometime perhaps we will know more about them - we
think. But that is just what commits us to a particular way of looking
at the matter." Thanks to this wonderful book, now is the time when we
will know more about processes and states of the brain.


\vspace{\baselineskip}
\begin{flushright}\noindent
Oxford Mathematical Institute\hfill {\it Alain Goriely}\\
January, 2021\hfill $\,$ \\ 
\end{flushright}
