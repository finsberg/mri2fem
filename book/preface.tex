%%%%%%%%%%%%%%%%%%%%%%preface.tex%%%%%%%%%%%%%%%%%%%%%%%%%%%%%%%%%%%%%%%%%
% sample preface
%
% Use this file as a template for your own input.
%
%%%%%%%%%%%%%%%%%%%%%%%% Springer %%%%%%%%%%%%%%%%%%%%%%%%%%

\preface

Observations surrounding the nature and fundamental biology of
humankind date back to some of our earliest written historical
accounts. Brain pulsatile behaviour and the structure of brain folding
were described in the ancient Egyptian \textit{Edwin Smith Surgical
Papyrus}\footnote{The text is named after the dealer who bought it.}
dating back to 1700 BC. Hippocrates, the father of medicine,
hypothesized that the brain was the \textit{seat of intelligence},
while Aristotle was fascinated by and wrote about both \textit{sleep}
and \textit{dreams}. However, early methods of directly investigating
human anatomy were crude and invasive.  Arguably, one of the profound
medical achievements of our modern age is the advent of non-invasive
imaging technologies.

The imaging revolution was born with Wilhelm R{\"o}ntgen's discovery
of X-rays as far back as in 1895. A few years later, Marie Curie
successfully isolated radium, and X-rays then began to be used
medically. Still, even with these early breakthroughs, positron
emission tomography (PET) would not arrive until 1950, and it would be
1967 before the first clinical X-ray computed tomography (CT) scanner
was put to use. An explosive development of different techniques
followed these successes and the reader who is interested in the
fascinating history of radiology can find more details
in \cite{thomas2013}. We now have many new imaging methods at our
disposal; in addition to PET and CT, the various magnetic resonance
(MR) imaging techniques\footnote{MR images will play a fundamental
role in this book. We introduce MR images in Chapter~\ref{chp:chp2},
but do take a sneak peek at
Figures~\ref{fig:chp2:brain}--\ref{fig:chp2:t1vt2}. It is astonishing
that, in less than 100 years since Isidor Rabi published his seminal
work measuring the nuclear spin of molecules \cite{thomas2013}, MRI
has become a cornerstone of medical science and of mathematical
modeling of the human brain.} have proven to be indispensable for
understanding the brain of living patients. The medical imaging field
is now vast. For instance, the annual meeting of the Radiological
Society of North America hosts around 25,000 attendees, while 44,000
people attended the meeting of the Society for Neuroscience.

Meanwhile, engineering and industrial applications have led to the
rapid development of both numerical methods, and applications using
partial differential equations (PDEs) to model and understand physical
phenomena. The finite element method (FEM), in particular, was
introduced in the 1960s for solving PDEs on domains with complex
geometries. Significant work has been invested in the construction of
scalable, performant, and approachable software libraries for solving
PDEs via the FEM. Today, we have many such libraries at our disposal;
including Abaqus, COMSOL, deal.II, DUNE, and {\fenics}. However, the
generation of suitable, physiological finite element geometries for
solving PDEs over brain domains remains a practical
barrier. Therefore, the impact of computational modeling on medical
imaging and neuroscience has not yet reached its full potential. Our
aim with this book is to provide a bridge between common tools in
medical imaging and neuroscience, and the numerical solution of PDEs
that can arise in brain modeling. More specifically, our work focuses
on the use of two existing tools, {\freesurfer} and {\fenics}, and one
novel tool, the {\svmtk}, developed for this book.
 
A central, and practical, problem preventing a more widespread
interest in the mathematical modeling of the human brain is that of
anatomical mesh generation.  Generating physiological finite element
meshes of the brain is not an easy task. The sulcal and gyral folds of
the cortex are intricate, and the extracellular diffusion tensor,
dictated largely by axonal white-matter bundles, is anisotropic and
tortuous. Nevertheless, such features are essential for even the
simplest, patient-specific PDE models of brain structural deformation
and fluid dynamics. The ability to accurately capture anatomical
features could help us address many human problems, particularly when
it comes to understanding the mechanisms underlying neurodegenerative
pathology evolution. This book stands at the gateway of these pressing
problems.

Herein, we guide the reader through a straightforward methodology for
ascertaining the basic assets, that is, a finite element mesh and the
extracellular diffusion tensor, from a set of patient MR images. To do
so, we introduce a novel software library, the Surface Volume Meshing
Toolkit (SVM-Tk), wrapping functionality from the broad array of
capabilities provided by the Computational Geometry Algorithms Library
(CGAL), thus offering an approachable set of features specifically for
the brain modeling community. Along the way, we will also employ the
automatic segmentation capacity of the FreeSurfer tool set, a
gold standard for MR image processing. The marriage of mathematical
modeling, clinical imaging, and numerical analysis is demonstrated by
solving a simplified PDE model of anisotropic gadobutrol diffusion in
the brain.

We are deeply grateful to the numerous colleagues who have provided
advice, and guidance, along the way as we commence our journey with
you, the reader, into the exciting world of mathematical brain
modeling. In particular, we thank Siri Fl\o gstad Svensson and Kyrre
Eeg Emblem for the imaging data, Johannes Ring for creating Docker
images, and Ana Budisa, J\o rgen Dokken, Kyrre Eeg Emblem, Ingeborg
Gjerde, Martin Hornkj\o l, Miroslav Kuchta, Yngve Mardal Moe, Geir
Ringstad, Vegard Vinje, and Bastian Zapf for their extremely
constructive feedback on the book and the associated software.  Jacob
Schreiner has made significant contributions to SVM-Tk and finally,
the we wish to thank Anders M. Dale for hosting several research
visits to La Jolla, which were instrumental for starting this project.


\vspace{\baselineskip}
\begin{flushright}\noindent
Oslo, Norway \hfill {\it Kent-Andr\'e Mardal}\\ 
Oxford, United Kingdom   \hfill {\it Marie E. Rognes}\\ 
             \hfill {\it Travis B. Thompson}\\ 
Jan, 2021    \hfill {\it Lars Magnus Valnes}\\ 
\end{flushright}


